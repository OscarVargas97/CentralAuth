\documentclass{article}
\usepackage[utf8]{inputenc}
\usepackage{tikz}

\title{CentralAuth: Microservicio Centralizado de Autenticación y Gestión de Usuarios}
\author{Oscar Andres Vargas Zazzali}
\date{25 de junio de 2024}

\begin{document}

\maketitle

\begin{center}
    \textbf{Versión del Documento: 1.0}
\end{center}
\newpage
\tableofcontents
\newpage

\section{Resumen Ejecutivo}

El proyecto \textbf{CentralAuth} consiste en desarrollar un microservicio centralizado para la autenticación y gestión de usuarios. Este microservicio permitirá la centralización de los datos de usuarios, mejorando la seguridad y eficiencia en la gestión de identidades. Además, ofrecerá funcionalidades como el registro de usuarios, inicio de sesión, recuperación de contraseñas y gestión de roles y permisos. El objetivo es proporcionar una solución escalable y reutilizable que pueda integrarse fácilmente en múltiples aplicaciones y servicios.

\subsection{Objetivos del Proyecto}

\begin{itemize}
    \item \textbf{Objetivo General:} Desarrollar un microservicio centralizado para la autenticación y gestión de usuarios, que ofrezca seguridad, escalabilidad y eficiencia en la administración de identidades.
    \item \textbf{Objetivos Específicos:}
          \begin{itemize}
              \item Centralización de Datos: Unificar la gestión de usuarios y autenticación en un único sistema centralizado para facilitar el mantenimiento y la administración.
              \item Seguridad: Implementar medidas de seguridad avanzadas, incluyendo cifrado de contraseñas, autenticación de dos factores (2FA) y protección contra ataques comunes como el phishing y el brute force.
              \item Escalabilidad: Diseñar el sistema de manera que pueda escalar horizontalmente para manejar un gran número de usuarios y solicitudes simultáneas.
              \item Interoperabilidad: Garantizar que el microservicio pueda integrarse fácilmente con diversas aplicaciones y plataformas mediante API RESTful.
              \item Gestión de Roles y Permisos: Proveer funcionalidades avanzadas para la gestión de roles y permisos, permitiendo una administración granular de las autorizaciones de usuario.
              \item Experiencia de Usuario: Asegurar una experiencia de usuario intuitiva y fluida tanto para los administradores como para los usuarios finales.
          \end{itemize}
\end{itemize}

\subsection{Importancia del Proyecto}

El proyecto \textbf{CentralAuth} es crucial por varias razones:

\begin{itemize}
    \item \textbf{Seguridad Mejorada:} Al centralizar la autenticación y gestión de usuarios, se pueden implementar medidas de seguridad robustas de manera consistente en todas las aplicaciones y servicios que utilicen el microservicio. Esto reduce el riesgo de brechas de seguridad y garantiza una protección uniforme de los datos de los usuarios.
    \item \textbf{Eficiencia Operativa:} Centralizar la administración de usuarios permite un manejo más eficiente de las identidades, roles y permisos. Los administradores pueden gestionar todos los usuarios desde un único punto, reduciendo la redundancia y el esfuerzo manual asociado con la gestión de múltiples sistemas de autenticación.
    \item \textbf{Escalabilidad y Flexibilidad:} Al ser un microservicio independiente, \textbf{CentralAuth} puede escalarse horizontalmente para manejar un aumento en la carga de trabajo, lo que es esencial para aplicaciones que experimentan un crecimiento rápido en el número de usuarios.
    \item \textbf{Reutilización y Consistencia:} Este microservicio puede ser reutilizado por diferentes aplicaciones dentro de una organización, asegurando que todas ellas sigan las mismas políticas de autenticación y gestión de usuarios. Esto promueve la consistencia y facilita el cumplimiento de normativas y estándares de seguridad.
    \item \textbf{Mejora de la Experiencia del Usuario:} Al ofrecer una autenticación rápida y segura, \textbf{CentralAuth} mejora la experiencia del usuario final, lo que puede aumentar la satisfacción y retención de usuarios en las aplicaciones que integren este servicio.
\end{itemize}

\subsection{Resumen de los Puntos Principales del Documento}

\begin{itemize}
    \item \textbf{Visión del Proyecto:} Descripción inspiradora del objetivo a largo plazo del proyecto y cómo contribuirá al éxito de la organización.
    \item \textbf{Introducción:} Contexto del proyecto, objetivos generales y específicos, y alcance del proyecto.
    \item \textbf{Descripción del Proyecto:} Detalles técnicos del proyecto, funcionalidades clave, tecnologías a utilizar y arquitectura del sistema.
    \item \textbf{Planificación del Proyecto:} Cronograma detallado, fases del proyecto y entregables de cada fase.
    \item \textbf{Recursos y Roles:} Equipo del proyecto y sus responsabilidades, y recursos necesarios.
    \item \textbf{Análisis de Riesgos:} Identificación de riesgos potenciales y plan de mitigación y contingencia.
    \item \textbf{Requisitos del Sistema:} Requisitos funcionales, no funcionales y de hardware y software.
    \item \textbf{Diseño del Sistema:} Diagramas de flujo, de entidad-relación, de clases y arquitectura de la base de datos.
    \item \textbf{Plan de Desarrollo:} Metodología de desarrollo, estrategias de implementación, plan de integración y pruebas.
    \item \textbf{Plan de Pruebas:} Tipos de pruebas a realizar, casos de prueba y plan de gestión de defectos.
    \item \textbf{Plan de Implementación:} Estrategias de despliegue, plan de capacitación para usuarios finales y plan de soporte post-implementación.
    \item \textbf{Plan de Mantenimiento:} Estrategias de mantenimiento y actualización, y plan de gestión de cambios.
    \item \textbf{Conclusiones:} Resumen de los puntos clave y próximos pasos.
    \item \textbf{Anexos:} Documentación adicional y referencias.
\end{itemize}

\newpage
\section{Visión del Proyecto}

\subsection{Descripción Inspiradora del Objetivo a Largo Plazo del Proyecto}

La visión de \textbf{CentralAuth} es convertirse en una opción popular para la autenticación y gestión de usuarios dentro de la comunidad de desarrolladores. Aspiramos a crear un sistema que centralice la gestión de identidades de manera eficiente y segura, ofreciendo escalabilidad y flexibilidad para satisfacer las necesidades de diversas aplicaciones y servicios. Nuestro objetivo es proporcionar una solución confiable y accesible que sea ampliamente adoptada por desarrolladores y organizaciones, facilitando la administración de usuarios y mejorando la seguridad en el acceso a sistemas.

\subsection{Cómo el Proyecto Contribuirá al Éxito de la Organización}

\textbf{CentralAuth} contribuirá al éxito de la organización al proporcionar una solución centralizada y eficiente para la gestión de usuarios y la autenticación. Esto permitirá a las organizaciones reducir los costos asociados con la implementación y el mantenimiento de múltiples sistemas de autenticación, mejorar la seguridad de los datos de usuario y garantizar una experiencia de usuario coherente en todas sus aplicaciones y servicios. Además, al facilitar la integración con diversas plataformas, \textbf{CentralAuth} permitirá a las organizaciones adaptarse rápidamente a nuevas tecnologías y demandas del mercado, fomentando la innovación y el crecimiento.

\subsection{Impacto Esperado y Beneficios a Largo Plazo}

\textbf{CentralAuth} se espera que tenga un impacto significativo en la eficiencia operativa y la seguridad de las organizaciones que lo adopten. A largo plazo, los beneficios incluyen:

\begin{itemize}
    \item \textbf{Reducción de Costos:} Menores costos de implementación y mantenimiento al centralizar la gestión de autenticación y usuarios en un solo sistema.
    \item \textbf{Mejora en la Seguridad:} Mayor protección de los datos de usuario mediante la implementación de medidas de seguridad avanzadas y la reducción de vulnerabilidades asociadas con la gestión dispersa de identidades.
    \item \textbf{Flexibilidad y Escalabilidad:} Capacidad para escalar horizontalmente y adaptarse a las necesidades crecientes de usuarios y aplicaciones, garantizando un rendimiento óptimo.
    \item \textbf{Facilidad de Integración:} Integración sencilla con múltiples aplicaciones y plataformas, facilitando la expansión y la adopción de nuevas tecnologías.
    \item \textbf{Consistencia en la Experiencia del Usuario:} Provisión de una experiencia de usuario coherente y segura en todas las aplicaciones y servicios de la organización.
\end{itemize}
\newpage

\section{Introducción}

\subsection{Contexto del Proyecto}

En la actualidad, muchas organizaciones enfrentan desafíos relacionados con la gestión de usuarios y la autenticación debido a la existencia de múltiples sistemas independientes. Esta fragmentación genera problemas de seguridad, inconsistencias en la experiencia del usuario y un aumento en los costos operativos. Además, la integración de nuevas aplicaciones y servicios a menudo se complica por la falta de un sistema centralizado y eficiente para la gestión de identidades. "CentralAuth" surge como una solución a estos problemas, proporcionando un microservicio centralizado que simplifica y fortalece la autenticación y gestión de usuarios.

\subsection{Objetivos del Proyecto}

\textbf{Objetivo General:}

Desarrollar un microservicio centralizado para la autenticación y gestión de usuarios, que ofrezca seguridad, escalabilidad y eficiencia en la administración de identidades.

\textbf{Objetivos Específicos:}

\begin{itemize}
    \item Centralización de Datos: Unificar la gestión de usuarios y autenticación en un único sistema centralizado para facilitar el mantenimiento y la administración.
    \item Seguridad: Implementar medidas de seguridad avanzadas, incluyendo cifrado de contraseñas y autenticación de dos factores (2FA).
    \item Escalabilidad: Diseñar el sistema de manera que pueda escalar horizontalmente para manejar un gran número de usuarios y solicitudes simultáneas.
    \item Interoperabilidad: Garantizar que el microservicio pueda integrarse fácilmente con diversas aplicaciones y plataformas mediante API RESTful.
    \item Experiencia de Usuario: Asegurar una experiencia de usuario intuitiva y fluida tanto para los administradores como para los usuarios finales.
\end{itemize}

\subsection{Alcance del Proyecto}

El alcance del proyecto "CentralAuth" abarca las siguientes actividades:
\begin{itemize}
    \item Diseño del Sistema: Crear la arquitectura del sistema y definir los componentes clave.
    \item Desarrollo del Microservicio: Implementar las funcionalidades principales.
    \item Pruebas: Realizar pruebas unitarias, de integración y de aceptación para asegurar la calidad del microservicio.
    \item Documentación: Documentar la API y proporcionar guías de integración para desarrolladores.
    \item Despliegue: Implementar el microservicio en un entorno de producción y asegurar su disponibilidad.
    \item Mantenimiento Inicial: Proveer soporte inicial y realizar ajustes necesarios después del despliegue.
\end{itemize}

\subsection{Alcance del Producto}

El alcance del producto "CentralAuth" incluye las siguientes funcionalidades principales:
\begin{itemize}
    \item Registro de Usuarios: Permitir a los usuarios crear una cuenta con su información básica.
    \item Inicio de Sesión: Autenticar a los usuarios mediante credenciales (correo electrónico y contraseña).
    \item Recuperación de Contraseñas: Permitir a los usuarios restablecer su contraseña mediante un correo electrónico de recuperación.
    \item Generación de Tokens JWT: Emitir tokens JWT para la autenticación segura de los usuarios en las aplicaciones conectadas.
    \item Verificación de Correo Electrónico: Enviar correos electrónicos de verificación para confirmar la dirección de correo electrónico de los usuarios.
\end{itemize}

\newpage

\section{Detalles Técnicos del Proyecto}

\subsection{Estructura Técnica General}

\subsubsection{API Gateway}
\begin{itemize}
    \item \textbf{Funciones:}
          \begin{itemize}
              \item Autenticación inicial de solicitudes HTTP.
              \item Enrutamiento de solicitudes a los microservicios correspondientes.
              \item Implementación de políticas de limitación de tasa para prevenir abusos.
          \end{itemize}
    \item \textbf{Tecnologías:}
          \begin{itemize}
              \item Laravel Passport o Laravel Sanctum para la gestión de tokens y autenticación OAuth.
              \item NGINX como servidor web.
              \item Redis para almacenamiento en caché de tokens y limitación de tasa.
          \end{itemize}
\end{itemize}

\subsubsection{Servicio de Autenticación}
\begin{itemize}
    \item \textbf{Funciones:}
          \begin{itemize}
              \item Manejo de inicio de sesión y cierre de sesión.
              \item Generación, validación y renovación de tokens JWT.
              \item Emisión de eventos cuando hay cambios en los usuarios, roles o permisos.
              \item Provisión de servicios de autenticación para otros sistemas, como monolitos o aplicaciones externas.
          \end{itemize}
    \item \textbf{Tecnologías:}
          \begin{itemize}
              \item Laravel Passport para OAuth y JWT.
              \item Base de datos PostgreSQL para almacenamiento de usuarios y tokens.
              \item Redis para almacenamiento en caché y gestión de colas.
              \item Laravel Event Broadcasting para emisión de eventos.
              \item WebSockets y Redis para transmisión de eventos en tiempo real.
          \end{itemize}
\end{itemize}

\subsubsection{Servicio de Gestión de Usuarios}
\begin{itemize}
    \item \textbf{Funciones:}
          \begin{itemize}
              \item CRUD (Crear, Leer, Actualizar, Eliminar) de cuentas de usuario.
              \item Validación y normalización de datos de usuario.
          \end{itemize}
    \item \textbf{Tecnologías:}
          \begin{itemize}
              \item Laravel Eloquent para ORM.
              \item PostgreSQL como base de datos principal.
              \item Laravel Valet para desarrollo local y pruebas.
          \end{itemize}
\end{itemize}

\subsubsection{Servicio de Roles y Permisos}
\begin{itemize}
    \item \textbf{Funciones:}
          \begin{itemize}
              \item Asignación y verificación de roles y permisos de usuario.
              \item Control de acceso basado en roles (RBAC).
          \end{itemize}
    \item \textbf{Tecnologías:}
          \begin{itemize}
              \item Laravel Permission (paquete Spatie) para la gestión de roles y permisos.
              \item PostgreSQL para almacenamiento de roles y permisos.
          \end{itemize}
\end{itemize}

\subsubsection{Bus de Eventos}
\begin{itemize}
    \item \textbf{Funciones:}
          \begin{itemize}
              \item Comunicación asincrónica entre microservicios.
              \item Emisión y suscripción a eventos relevantes del sistema.
          \end{itemize}
    \item \textbf{Tecnologías:}
          \begin{itemize}
              \item Laravel Event Broadcasting con Redis y WebSockets para la transmisión de eventos.
              \item Supervisor para la gestión de colas y trabajos en segundo plano.
          \end{itemize}
\end{itemize}

\subsubsection{Base de Datos Central}
\begin{itemize}
    \item \textbf{Funciones:}
          \begin{itemize}
              \item Almacenamiento de datos relacionados con usuarios, roles y permisos.
              \item Garantía de integridad y consistencia de datos.
          \end{itemize}
    \item \textbf{Tecnologías:}
          \begin{itemize}
              \item PostgreSQL como base de datos principal.
              \item Configuración de replicación y clustering para alta disponibilidad.
          \end{itemize}
\end{itemize}

\subsection{Interacción entre Componentes}

\subsubsection{Flujo de Solicitudes HTTP}
\begin{itemize}
    \item El \textbf{API Gateway} (NGINX) recibe las solicitudes y las autentica con el \textbf{Servicio de Autenticación} (usando Laravel Passport). Luego, enruta las solicitudes autenticadas al microservicio adecuado.
    \item \textbf{Servicio de Autenticación}: Verifica credenciales y genera tokens JWT. Utiliza Laravel Passport para la gestión de tokens.
    \item \textbf{Servicio de Gestión de Usuarios}: Procesa operaciones CRUD interactuando directamente con la \textbf{Base de Datos Central}.
\end{itemize}

\subsubsection{Control de Acceso}
\begin{itemize}
    \item \textbf{Servicio de Roles y Permisos}: Verifica permisos en cada solicitud utilizando Laravel Permission, asegurando que el usuario tenga el acceso adecuado. Este servicio se comunica constantemente con el \textbf{Servicio de Gestión de Usuarios} para asegurar que cualquier cambio en los roles y permisos se refleje en tiempo real.
    \item \textbf{Bus de Eventos}: Facilita la emisión de eventos cuando se crean, actualizan o eliminan usuarios y sus roles/permisos, permitiendo que otras aplicaciones se actualicen en tiempo real mediante Laravel Event Broadcasting.
\end{itemize}

\subsubsection{Comunicación Asincrónica}
\begin{itemize}
    \item \textbf{Bus de Eventos}: Utiliza Redis y WebSockets para permitir que las aplicaciones se suscriban a eventos relevantes y manejar datos de usuarios eficientemente. Los eventos incluyen, pero no se limitan a, la creación, modificación y eliminación de usuarios y sus roles/permisos.
\end{itemize}

\subsection{Integración con Otros Sistemas}

El Servicio de Autenticación no solo se utiliza para los microservicios dentro del sistema, sino que también proporciona servicios de autenticación y autorización para sistemas externos, como un monolito o aplicaciones de terceros. Esto se logra mediante la provisión de endpoints de autenticación que permiten a estos sistemas:
\begin{itemize}
    \item Validar tokens JWT emitidos por el Servicio de Autenticación.
    \item Consultar roles y permisos de usuarios para implementar control de acceso interno.
    \item Integrarse fácilmente mediante API RESTful, asegurando que cualquier sistema pueda verificar la autenticidad de los usuarios y sus permisos antes de permitir el acceso a recursos internos.
\end{itemize}

\subsection{Ampliación de la Interacción entre Componentes}

Para asegurar una comunicación efectiva y la actualización en tiempo real de los datos, se han implementado varios patrones de diseño y tecnologías:
\begin{itemize}
    \item \textbf{Patrón de Saga}: Para gestionar transacciones distribuidas entre los servicios, asegurando la consistencia eventual.
    \item \textbf{CQRS (Command Query Responsibility Segregation)}: Para separar las operaciones de lectura y escritura, optimizando el rendimiento.
    \item \textbf{Event Sourcing}: Para registrar todos los cambios de estado como una secuencia de eventos, facilitando el monitoreo y la auditoría.
\end{itemize}

\subsection{Escalabilidad y Disponibilidad}

Cada componente del sistema está diseñado para ser escalable de manera independiente:
\begin{itemize}
    \item \textbf{API Gateway} y \textbf{Servicios de Autenticación}: Pueden escalar horizontalmente mediante balanceadores de carga.
    \item \textbf{Base de Datos Central}: Configurada con replicación para alta disponibilidad y recuperación ante fallos.
\end{itemize}

\subsection{Seguridad}

Se han implementado varias capas de seguridad para proteger los datos y las comunicaciones:
\begin{itemize}
    \item \textbf{Cifrado de datos en tránsito y en reposo}.
    \item \textbf{Autenticación multifactor (MFA)} para acceso de usuarios sensibles.
    \item \textbf{Auditorías y registros de actividad} para monitorear accesos y cambios.
\end{itemize}


\usetikzlibrary{shapes.geometric, arrows}

\tikzstyle{arrow} = [thick,->,>=stealth]


\end{document}
