
\section{Descripción del Proyecto}

\subsection{Detalles Técnicos del Proyecto}

\subsubsection{Estructura Técnica General}

\paragraph{API Gateway}
\begin{itemize}
    \item \textbf{Funciones:}
          \begin{itemize}
              \item Autenticación inicial de solicitudes HTTP.
              \item Enrutamiento de solicitudes a los microservicios correspondientes.
              \item Implementación de políticas de limitación de tasa para prevenir abusos.
          \end{itemize}
    \item \textbf{Tecnologías:}
          \begin{itemize}
              \item Laravel Passport o Laravel Sanctum para la gestión de tokens y autenticación OAuth.
              \item NGINX como servidor web.
              \item Redis para almacenamiento en caché de tokens y limitación de tasa.
          \end{itemize}
\end{itemize}

\paragraph{Servicio de Autenticación}
\begin{itemize}
    \item \textbf{Funciones:}
          \begin{itemize}
              \item Manejo de inicio de sesión y cierre de sesión.
              \item Generación, validación y renovación de tokens JWT.
              \item Emisión de eventos de desautenticación global.
              \item Provisión de servicios de autenticación para otros sistemas, como monolitos o aplicaciones externas.
          \end{itemize}
    \item \textbf{Tecnologías:}
          \begin{itemize}
              \item Laravel Passport para OAuth y JWT.
              \item Base de datos PostgreSQL para almacenamiento de usuarios y tokens.
              \item Redis para almacenamiento en caché y gestión de colas.
              \item Laravel Event Broadcasting para emisión de eventos.
              \item WebSockets y Redis para transmisión de eventos en tiempo real.
          \end{itemize}
\end{itemize}

\paragraph{Servicio de Gestión de Usuarios}
\begin{itemize}
    \item \textbf{Funciones:}
          \begin{itemize}
              \item CRUD (Crear, Leer, Actualizar, Eliminar) de cuentas de usuario.
              \item Validación y normalización de datos de usuario.
          \end{itemize}
    \item \textbf{Tecnologías:}
          \begin{itemize}
              \item Laravel Eloquent para ORM.
              \item PostgreSQL como base de datos principal.
              \item Laravel Valet para desarrollo local y pruebas.
          \end{itemize}
\end{itemize}

\paragraph{Bus de Eventos}
\begin{itemize}
    \item \textbf{Funciones:}
          \begin{itemize}
              \item Comunicación asincrónica entre microservicios.
              \item Emisión y suscripción a eventos relevantes del sistema.
              \item Manejo de eventos de desautenticación global para invalidar sesiones en todas las plataformas.
          \end{itemize}
    \item \textbf{Tecnologías:}
          \begin{itemize}
              \item Laravel Event Broadcasting con Redis y WebSockets para la transmisión de eventos.
              \item Supervisor para la gestión de colas y trabajos en segundo plano.
          \end{itemize}
\end{itemize}

\paragraph{Base de Datos Central}
\begin{itemize}
    \item \textbf{Funciones:}
          \begin{itemize}
              \item Almacenamiento de datos relacionados con usuarios y autenticación.
              \item Garantía de integridad y consistencia de datos.
          \end{itemize}
    \item \textbf{Tecnologías:}
          \begin{itemize}
              \item PostgreSQL como base de datos principal.
              \item Configuración de replicación y clustering para alta disponibilidad.
          \end{itemize}
\end{itemize}

\subsubsection{Interacción entre Componentes}
\begin{itemize}
    \item \textbf{Flujo de Solicitudes HTTP:}
          \begin{itemize}
              \item El \textbf{API Gateway} (NGINX) recibe las solicitudes y las autentica con el \textbf{Servicio de Autenticación} (usando Laravel Passport). Luego, enruta las solicitudes autenticadas al microservicio adecuado.
              \item \textbf{Servicio de Autenticación}: Verifica credenciales y genera tokens JWT. Utiliza Laravel Passport para la gestión de tokens.
              \item \textbf{Servicio de Gestión de Usuarios}: Procesa operaciones CRUD interactuando directamente con la \textbf{Base de Datos Central}.
          \end{itemize}
    \item \textbf{Control de Acceso:}
          \begin{itemize}
              \item \textbf{Bus de Eventos}: Facilita la emisión de eventos cuando se crean, actualizan o eliminan usuarios, permitiendo que otras aplicaciones se actualicen en tiempo real mediante Laravel Event Broadcasting.
          \end{itemize}
    \item \textbf{Comunicación Asincrónica:}
          \begin{itemize}
              \item \textbf{Bus de Eventos}: Utiliza Redis y WebSockets para permitir que las aplicaciones se suscriban a eventos relevantes y manejar datos de usuarios eficientemente. Los eventos incluyen, pero no se limitan a, la creación, modificación y eliminación de usuarios.
          \end{itemize}
\end{itemize}

\subsubsection{Integración con Otros Sistemas}
\begin{itemize}
    \item El Servicio de Autenticación no solo se utiliza para los microservicios dentro del sistema, sino que también proporciona servicios de autenticación y autorización para sistemas externos, como un monolito o aplicaciones de terceros. Esto se logra mediante la provisión de endpoints de autenticación que permiten a estos sistemas:
          \begin{itemize}
              \item Validar tokens JWT emitidos por el Servicio de Autenticación.
              \item Consultar roles y permisos de usuarios para implementar control de acceso interno.
              \item Integrarse fácilmente mediante API RESTful, asegurando que cualquier sistema pueda verificar la autenticidad de los usuarios y sus permisos antes de permitir el acceso a recursos internos.
          \end{itemize}
\end{itemize}

\subsubsection{Ampliación de la Interacción entre Componentes}
\begin{itemize}
    \item Para asegurar una comunicación efectiva y la actualización en tiempo real de los datos, se han implementado varios patrones de diseño y tecnologías:
          \begin{itemize}
              \item \textbf{Patrón de Saga}: Para gestionar transacciones distribuidas entre los servicios, asegurando la consistencia eventual.
              \item \textbf{CQRS (Command Query Responsibility Segregation)}: Para separar las operaciones de lectura y escritura, optimizando el rendimiento.
              \item \textbf{Event Sourcing}: Para registrar todos los cambios de estado como una secuencia de eventos, facilitando el monitoreo y la auditoría.
          \end{itemize}
\end{itemize}

\subsubsection{Escalabilidad y Disponibilidad}
\begin{itemize}
    \item Cada componente del sistema está diseñado para ser escalable de manera independiente:
          \begin{itemize}
              \item \textbf{API Gateway} y \textbf{Servicios de Autenticación}: Pueden escalar horizontalmente mediante balanceadores de carga.
              \item \textbf{Base de Datos Central}: Configurada con replicación para alta disponibilidad y recuperación ante fallos.
          \end{itemize}
\end{itemize}

\subsubsection{Seguridad}
\begin{itemize}
    \item Se han implementado varias capas de seguridad para proteger los datos y las comunicaciones:
          \begin{itemize}
              \item \textbf{Cifrado de datos en tránsito y en reposo}.
              \item \textbf{Autenticación multifactor (MFA)} para acceso de usuarios sensibles.
              \item \textbf{Auditorías y registros de actividad} para monitorear accesos y cambios.
          \end{itemize}
\end{itemize}

\subsection{Funcionalidades Clave}

\begin{itemize}
    \item \textbf{Registro de Usuarios:} Permitir a las aplicaciones cliente registrar nuevos usuarios en el sistema.
          \begin{itemize}
              \item Los clientes pueden enviar la información del usuario, como nombre, correo electrónico y contraseña, y recibir una confirmación del registro.
          \end{itemize}
    \item \textbf{Inicio de Sesión:} Permitir a las aplicaciones cliente autenticar a los usuarios existentes.
          \begin{itemize}
              \item Las aplicaciones pueden enviar las credenciales del usuario y recibir una confirmación de autenticación si las credenciales son correctas.
          \end{itemize}
    \item \textbf{Recuperación de Contraseñas:} Proveer una función para que las aplicaciones cliente inicien el proceso de recuperación de contraseña.
          \begin{itemize}
              \item Las aplicaciones pueden solicitar el restablecimiento de la contraseña y enviar un enlace de recuperación al correo electrónico del usuario.
          \end{itemize}
    \item \textbf{Generación de Tokens JWT:} Proveer una función para emitir y renovar tokens JWT para la autenticación segura de los usuarios.
          \begin{itemize}
              \item Las aplicaciones cliente pueden solicitar la emisión de tokens al iniciar sesión y la renovación de tokens cuando sea necesario.
          \end{itemize}
    \item \textbf{Verificación de Correo Electrónico:} Proveer una función para que las aplicaciones cliente envíen correos electrónicos de verificación a los usuarios.
          \begin{itemize}
              \item Los clientes pueden iniciar el envío de correos de verificación y confirmar la verificación de la dirección de correo electrónico del usuario.
          \end{itemize}
    \item \textbf{Autenticación Multifactor (2FA):} Proveer una función para que las aplicaciones cliente activen y gestionen la autenticación de dos factores para los usuarios.
          \begin{itemize}
              \item Las aplicaciones pueden habilitar el 2FA, enviar códigos de verificación y validar estos códigos para el acceso seguro.
          \end{itemize}
    \item \textbf{Gestión de Sesiones:} Proveer una función para que las aplicaciones cliente monitoricen y gestionen las sesiones de usuario.
          \begin{itemize}
              \item Los clientes pueden consultar sesiones activas, cerrar sesiones y gestionar la duración de las sesiones.
          \end{itemize}
    \item \textbf{Interoperabilidad con Otras Aplicaciones:} Asegurar que el microservicio puede integrarse fácilmente con diversas aplicaciones.
          \begin{itemize}
              \item Ofrecer endpoints estándar para que las aplicaciones puedan autenticar usuarios y consultar roles y permisos específicos de cada aplicación.
          \end{itemize}
    \item \textbf{Desautenticación Global:} Permitir a las aplicaciones cliente cerrar la sesión de un usuario en todas las plataformas de manera coordinada.
          \begin{itemize}
              \item Emitir eventos de desautenticación global que invaliden sesiones en todas las aplicaciones suscritas.
          \end{itemize}
\end{itemize}

\subsection{Tecnologías a Utilizar}

\subsubsection{Lenguajes de Programación}
\begin{itemize}
    \item \textbf{PHP:} Utilizado para desarrollar el backend del microservicio de autenticación.
    \item \textbf{JavaScript:} Utilizado para el desarrollo del frontend.
\end{itemize}

\subsubsection{Frameworks}
\begin{itemize}
    \item \textbf{Laravel:} Framework de PHP utilizado para el desarrollo del backend.
    \item \textbf{Vue.js:} Framework de JavaScript utilizado para el desarrollo del frontend.
\end{itemize}

\subsubsection{Bases de Datos}
\begin{itemize}
    \item \textbf{PostgreSQL:} Base de datos relacional utilizada para almacenar los datos de los usuarios y tokens de autenticación.
\end{itemize}

\subsubsection{Servidores y Hosting}
\begin{itemize}
    \item \textbf{NGINX:} Utilizado como servidor web y balanceador de carga.
    \item \textbf{AWS:} Servicios de Amazon Web Services para hosting y escalabilidad.
\end{itemize}

\subsubsection{Seguridad}
\begin{itemize}
    \item \textbf{Cifrado:} TLS/SSL para cifrado de datos en tránsito.
    \item \textbf{Autenticación Multifactor (2FA):} Implementación de 2FA para seguridad adicional.
    \item \textbf{OAuth y JWT:} Para la gestión de tokens y autenticación segura.
\end{itemize}

\subsubsection{Herramientas de Desarrollo y CI/CD}
\begin{itemize}
    \item \textbf{GitHub:} Repositorio de código y gestión de versiones.
    \item \textbf{Jenkins:} Herramienta de integración continua y despliegue continuo.
\end{itemize}

\subsubsection{Servicios de Mensajería}
\begin{itemize}
    \item \textbf{Redis:} Utilizado para la gestión de colas y almacenamiento en caché.
    \item \textbf{RabbitMQ:} Utilizado para la comunicación asincrónica entre microservicios.
\end{itemize}

\subsubsection{Monitoreo y Logging}
\begin{itemize}
    \item \textbf{ELK Stack:} ElasticSearch, Logstash y Kibana para monitoreo y análisis de logs.
    \item \textbf{Prometheus y Grafana:} Para monitoreo del rendimiento y visualización de métricas.
\end{itemize}

