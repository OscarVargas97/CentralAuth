\section{Análisis de Riesgos}

\subsection{Escala de Gravedad del Riesgo}

\subsubsection{Escala de Probabilidad}

\begin{itemize}
    \item \textbf{Baja (1)}: El riesgo es poco probable que ocurra.
    \item \textbf{Media (2)}: El riesgo tiene una posibilidad moderada de ocurrir.
    \item \textbf{Alta (3)}: El riesgo es muy probable que ocurra.
\end{itemize}

\subsubsection{Escala de Impacto}

\begin{itemize}
    \item \textbf{Bajo (1)}: El impacto del riesgo en el proyecto es mínimo y fácil de manejar.
    \item \textbf{Moderado (2)}: El impacto del riesgo puede causar problemas significativos pero manejables.
    \item \textbf{Alto (3)}: El impacto del riesgo puede ser devastador para el proyecto, causando grandes retrasos o fallos críticos.
\end{itemize}

\subsection{Identificación y Evaluación de Riesgos Potenciales}

\subsubsection{Riesgos del Proyecto}

\begin{itemize}
    \item \textbf{Cambio de Requisitos: Seguridad}
          \begin{itemize}
              \item \textbf{Descripción}: Los requisitos de seguridad del proyecto pueden cambiar debido a nuevas necesidades o descubrimientos de vulnerabilidades.
              \item \textbf{Probabilidad}: 2 (Media)
              \item \textbf{Impacto}: 3 (Alto)
          \end{itemize}

    \item \textbf{Cambio de Requisitos: Escalabilidad}
          \begin{itemize}
              \item \textbf{Descripción}: Los requisitos de escalabilidad del proyecto pueden cambiar debido a un aumento inesperado en la cantidad de usuarios.
              \item \textbf{Probabilidad}: 2 (Media)
              \item \textbf{Impacto}: 3 (Alto)
          \end{itemize}

    \item \textbf{Cambio de Requisitos: Integración}
          \begin{itemize}
              \item \textbf{Descripción}: Los requisitos de integración del proyecto pueden cambiar debido a la necesidad de soportar nuevas aplicaciones cliente.
              \item \textbf{Probabilidad}: 2 (Media)
              \item \textbf{Impacto}: 2 (Moderado)
          \end{itemize}

    \item \textbf{Falta de Comunicación con Partes Interesadas}
          \begin{itemize}
              \item \textbf{Descripción}: Problemas de comunicación pueden surgir con las partes interesadas, causando malentendidos y retrasos.
              \item \textbf{Probabilidad}: 2 (Media)
              \item \textbf{Impacto}: 2 (Moderado)
          \end{itemize}

    \item \textbf{Desgaste del Equipo por Alta Carga de Trabajo}
          \begin{itemize}
              \item \textbf{Descripción}: La alta carga de trabajo puede afectar la moral y la productividad de la persona a cargo.
              \item \textbf{Probabilidad}: 3 (Alta)
              \item \textbf{Impacto}: 3 (Alto)
          \end{itemize}

    \item \textbf{Desgaste del Equipo por Plazos Ajustados}
          \begin{itemize}
              \item \textbf{Descripción}: La presión para cumplir con plazos ajustados puede afectar la moral y la productividad de la persona a cargo.
              \item \textbf{Probabilidad}: 3 (Alta)
              \item \textbf{Impacto}: 2 (Moderado)
          \end{itemize}

    \item \textbf{Falta de Redundancia: Indisponibilidad del Personal}
          \begin{itemize}
              \item \textbf{Descripción}: La falta de personal adicional significa que si la persona a cargo está indisponible, el proyecto puede verse seriamente afectado.
              \item \textbf{Probabilidad}: 3 (Alta)
              \item \textbf{Impacto}: 3 (Alto)
          \end{itemize}
\end{itemize}

\subsubsection{Riesgos Técnicos}

\begin{itemize}
    \item \textbf{Problemas de Calidad: Pruebas Inadecuadas}
          \begin{itemize}
              \item \textbf{Descripción}: La falta de pruebas adecuadas puede resultar en la entrega de un microservicio con defectos.
              \item \textbf{Probabilidad}: 2 (Media)
              \item \textbf{Impacto}: 3 (Alto)
          \end{itemize}

    \item \textbf{Problemas de Calidad: Defectos en Autenticación}
          \begin{itemize}
              \item \textbf{Descripción}: Defectos en el manejo de la autenticación pueden comprometer la seguridad del sistema.
              \item \textbf{Probabilidad}: 2 (Media)
              \item \textbf{Impacto}: 3 (Alto)
          \end{itemize}

    \item \textbf{Problemas de Calidad: Defectos en Gestión de Usuarios}
          \begin{itemize}
              \item \textbf{Descripción}: Defectos en la gestión de usuarios pueden llevar a la pérdida o corrupción de datos.
              \item \textbf{Probabilidad}: 2 (Media)
              \item \textbf{Impacto}: 2 (Moderado)
          \end{itemize}

    \item \textbf{Problemas de Calidad: Defectos en API}
          \begin{itemize}
              \item \textbf{Descripción}: Defectos en las APIs pueden resultar en fallos de integración con las aplicaciones cliente.
              \item \textbf{Probabilidad}: 2 (Media)
              \item \textbf{Impacto}: 2 (Moderado)
          \end{itemize}

    \item \textbf{Problemas de Calidad: Defectos en Seguridad}
          \begin{itemize}
              \item \textbf{Descripción}: Defectos en la implementación de medidas de seguridad pueden dejar el sistema vulnerable a ataques.
              \item \textbf{Probabilidad}: 2 (Media)
              \item \textbf{Impacto}: 3 (Alto)
          \end{itemize}

    \item \textbf{Problemas de Compatibilidad de Software}
          \begin{itemize}
              \item \textbf{Descripción}: Problemas de compatibilidad entre diferentes versiones de software y bibliotecas utilizadas pueden causar fallos en el sistema.
              \item \textbf{Probabilidad}: 2 (Media)
              \item \textbf{Impacto}: 2 (Moderado)
          \end{itemize}

    \item \textbf{Complejidad del Código}
          \begin{itemize}
              \item \textbf{Descripción}: La programación orientada a eventos puede llevar a un código más complejo y difícil de mantener debido a la naturaleza asíncrona y las múltiples fuentes de eventos.
              \item \textbf{Probabilidad}: 2 (Media)
              \item \textbf{Impacto}: 2 (Moderado)
          \end{itemize}

    \item \textbf{Problemas de Debugging}
          \begin{itemize}
              \item \textbf{Descripción}: La depuración de aplicaciones orientadas a eventos puede ser más difícil debido a la falta de una secuencia clara de ejecución.
              \item \textbf{Probabilidad}: 2 (Media)
              \item \textbf{Impacto}: 2 (Moderado)
          \end{itemize}

    \item \textbf{Condiciones de Carrera}
          \begin{itemize}
              \item \textbf{Descripción}: La naturaleza asíncrona de la EDP puede introducir condiciones de carrera, donde dos o más eventos intentan acceder o modificar el mismo recurso al mismo tiempo.
              \item \textbf{Probabilidad}: 2 (Media)
              \item \textbf{Impacto}: 3 (Alto)
          \end{itemize}

    \item \textbf{Bloqueos y Errores Difíciles de Reproducir}
          \begin{itemize}
              \item \textbf{Descripción}: Los errores en aplicaciones orientadas a eventos pueden ser difíciles de reproducir y diagnosticar, especialmente cuando dependen de la interacción de múltiples eventos.
              \item \textbf{Probabilidad}: 2 (Media)
              \item \textbf{Impacto}: 3 (Alto)
          \end{itemize}
\end{itemize}

\subsubsection{Riesgos de Seguridad}

\begin{itemize}
    \item \textbf{Exposición de Datos Sensibles}
          \begin{itemize}
              \item \textbf{Descripción}: Fallos en la seguridad pueden resultar en la exposición de datos sensibles de los usuarios.
              \item \textbf{Probabilidad}: 2 (Media)
              \item \textbf{Impacto}: 3 (Alto)
          \end{itemize}

    \item \textbf{Ataques de Fuerza Bruta}
          \begin{itemize}
              \item \textbf{Descripción}: El sistema puede ser vulnerable a ataques de fuerza bruta en las contraseñas.
              \item \textbf{Probabilidad}: 2 (Media)
              \item \textbf{Impacto}: 3 (Alto)
          \end{itemize}

    \item \textbf{Phishing}
          \begin{itemize}
              \item \textbf{Descripción}: Usuarios pueden ser víctimas de ataques de phishing, comprometiendo sus credenciales.
              \item \textbf{Probabilidad}: 2 (Media)
              \item \textbf{Impacto}: 3 (Alto)
          \end{itemize}

    \item \textbf{Vulnerabilidades en Bibliotecas Externas}
          \begin{itemize}
              \item \textbf{Descripción}: Las bibliotecas externas pueden tener vulnerabilidades que pueden ser explotadas por atacantes.
              \item \textbf{Probabilidad}: 2 (Media)
              \item \textbf{Impacto}: 3 (Alto)
          \end{itemize}
\end{itemize}

\subsubsection{Riesgos de Escalabilidad y Rendimiento}

\begin{itemize}
    \item \textbf{Degradación del Rendimiento Bajo Alta Carga}
          \begin{itemize}
              \item \textbf{Descripción}: El sistema puede no manejar adecuadamente un aumento repentino en la carga de usuarios y solicitudes.
              \item \textbf{Probabilidad}: 3 (Alta)
              \item \textbf{Impacto}: 3 (Alto)
          \end{itemize}

    \item \textbf{Limitaciones de Escalabilidad Horizontal}
          \begin{itemize}
              \item \textbf{Descripción}: El diseño del sistema puede no soportar escalabilidad horizontal efectiva.
              \item \textbf{Probabilidad}: 2 (Media)
              \item \textbf{Impacto}: 3 (Alto)
          \end{itemize}

    \item \textbf{Problemas de Tiempo de Respuesta}
          \begin{itemize}
              \item \textbf{Descripción}: Aumento en el tiempo de respuesta bajo alta carga de usuarios puede afectar la experiencia del usuario.
              \item \textbf{Probabilidad}: 3 (Alta)
              \item \textbf{Impacto}: 3 (Alto)
          \end{itemize}

    \item \textbf{Sobrecarga de Eventos}
          \begin{itemize}
              \item \textbf{Descripción}: Un exceso de eventos puede sobrecargar el sistema, provocando una disminución del rendimiento o incluso fallos en el sistema.
              \item \textbf{Probabilidad}: 2 (Media)
              \item \textbf{Impacto}: 2 (Moderado)
          \end{itemize}

    \item \textbf{Latencia en el Procesamiento de Eventos}
          \begin{itemize}
              \item \textbf{Descripción}: La latencia en el procesamiento de eventos puede afectar la experiencia del usuario, especialmente en sistemas que requieren respuestas en tiempo real.
              \item \textbf{Probabilidad}: 2 (Media)
              \item \textbf{Impacto}: 2 (Moderado)
          \end{itemize}
\end{itemize}

\subsubsection{Riesgos de Integración}

\begin{itemize}
    \item \textbf{Incompatibilidad con APIs Existentes}
          \begin{itemize}
              \item \textbf{Descripción}: Incompatibilidades con las APIs existentes pueden causar problemas en la integración con aplicaciones cliente.
              \item \textbf{Probabilidad}: 2 (Media)
              \item \textbf{Impacto}: 2 (Moderado)
          \end{itemize}

    \item \textbf{Errores en la Comunicación entre Servicios}
          \begin{itemize}
              \item \textbf{Descripción}: Errores en la comunicación entre microservicios pueden llevar a fallos en el sistema.
              \item \textbf{Probabilidad}: 2 (Media)
              \item \textbf{Impacto}: 3 (Alto)
          \end{itemize}

    \item \textbf{Compatibilidad de Eventos entre Sistemas}
          \begin{itemize}
              \item \textbf{Descripción}: Problemas de compatibilidad entre los eventos generados por diferentes sistemas pueden causar errores en la integración.
              \item \textbf{Probabilidad}: 2 (Media)
              \item \textbf{Impacto}: 2 (Moderado)
          \end{itemize}

    \item \textbf{Manejo de Eventos No Previstos}
          \begin{itemize}
              \item \textbf{Descripción}: La aparición de eventos no previstos puede llevar a comportamientos inesperados o errores en el sistema.
              \item \textbf{Probabilidad}: 2 (Media)
              \item \textbf{Impacto}: 2 (Moderado)
          \end{itemize}
\end{itemize}

\subsubsection{Riesgos de Disponibilidad}

\begin{itemize}
    \item \textbf{Interrupciones en Servicios Externos}
          \begin{itemize}
              \item \textbf{Descripción}: Caídas o interrupciones en los servicios externos (por ejemplo, AWS, Redis) pueden afectar la disponibilidad del sistema.
              \item \textbf{Probabilidad}: 2 (Media)
              \item \textbf{Impacto}: 3 (Alto)
          \end{itemize}

    \item \textbf{Limitaciones de Capacidad en Servicios Externos}
          \begin{itemize}
              \item \textbf{Descripción}: Limitaciones en los servicios de nube utilizados pueden afectar la disponibilidad y escalabilidad del sistema.
              \item \textbf{Probabilidad}: 2 (Media)
              \item \textbf{Impacto}: 3 (Alto)
          \end{itemize}

    \item \textbf{Dependencia de Herramientas de CI/CD}
          \begin{itemize}
              \item \textbf{Descripción}: Problemas con herramientas de CI/CD pueden retrasar el despliegue y la integración continua.
              \item \textbf{Probabilidad}: 2 (Media)
              \item \textbf{Impacto}: 2 (Moderado)
          \end{itemize}

    \item \textbf{Dependencia de Laravel}
          \begin{itemize}
              \item \textbf{Descripción}: Problemas o actualizaciones incompatibles en Laravel pueden afectar el desarrollo y mantenimiento del backend.
              \item \textbf{Probabilidad}: 2 (Media)
              \item \textbf{Impacto}: 3 (Alto)
          \end{itemize}

    \item \textbf{Dependencia de Vue.js}
          \begin{itemize}
              \item \textbf{Descripción}: Problemas o actualizaciones incompatibles en Vue.js pueden afectar el desarrollo y mantenimiento del frontend.
              \item \textbf{Probabilidad}: 2 (Media)
              \item \textbf{Impacto}: 3 (Alto)
          \end{itemize}

    \item \textbf{Interrupciones de Servicio por Mantenimiento}
          \begin{itemize}
              \item \textbf{Descripción}: Mantenimiento no planificado puede causar interrupciones en el servicio.
              \item \textbf{Probabilidad}: 2 (Media)
              \item \textbf{Impacto}: 2 (Moderado)
          \end{itemize}

    \item \textbf{Caídas del Servidor}
          \begin{itemize}
              \item \textbf{Descripción}: Problemas de infraestructura pueden resultar en caídas del servidor y pérdida de disponibilidad del servicio.
              \item \textbf{Probabilidad}: 2 (Media)
              \item \textbf{Impacto}: 3 (Alto)
          \end{itemize}

    \item \textbf{Limitaciones de Capacidad en Servicios de Nube}
          \begin{itemize}
              \item \textbf{Descripción}: Limitaciones en los servicios de nube utilizados pueden afectar la disponibilidad y escalabilidad del sistema.
              \item \textbf{Probabilidad}: 2 (Media)
              \item \textbf{Impacto}: 3 (Alto)
          \end{itemize}
\end{itemize}

\subsubsection{Riesgos Legales y Regulatorios}

\begin{itemize}
    \item \textbf{Incumplimiento de Normativas de Privacidad de Datos}
          \begin{itemize}
              \item \textbf{Descripción}: Fallos en el cumplimiento de las normativas de privacidad de datos pueden resultar en sanciones.
              \item \textbf{Probabilidad}: 2 (Media)
              \item \textbf{Impacto}: 3 (Alto)
          \end{itemize}

    \item \textbf{Problemas de Propiedad Intelectual}
          \begin{itemize}
              \item \textbf{Descripción}: Uso no autorizado de bibliotecas o herramientas de software puede llevar a problemas de propiedad intelectual.
              \item \textbf{Probabilidad}: 2 (Media)
              \item \textbf{Impacto}: 3 (Alto)
          \end{itemize}

    \item \textbf{Incumplimiento de Licencias de Software}
          \begin{itemize}
              \item \textbf{Descripción}: Uso incorrecto o no autorizado de bibliotecas o herramientas de software puede llevar a problemas legales.
              \item \textbf{Probabilidad}: 2 (Media)
              \item \textbf{Impacto}: 3 (Alto)
          \end{itemize}
\end{itemize}
